% !Mode:: "TeX:UTF-8"
\chapter{简介}

\section{为什么}

“为什么用\LaTeX{}{}?”

当把\LaTeX{}{}介绍给一个人的时候,这是面对的第一个问题。
当回答“它很好用时”,又会得到第二个问题:
“Word不好用吗?”答案当然是肯定的,
Microsoft Word\index{Word}是这个世界上目前最流行的,最易用的文字处理软件之一。

但是,
我这里要做一个转折了,如果是做过比较大的文档的,几十页以上,有分章,
分节,或者还要有页眉页脚页码目录以及封面这些东西,好吧,大部分人做过唯一的这种文档,那就是毕业论文了。
给毕业论文排版,几乎是大部分人的一场梦魇。如果性格里再带一点点追求完美的
念头,追求一点点排版的质量,那给自己的毕业论文排版,那一定会留下一生深刻的印象!

页码格式不对,页码编号不对,不同章节标题字体不一致,小标号缩进老对不齐,
数字有时候是Times Roman字体,
有的时候又成了宋体,有的段落前有一块空白,有的段落间距又太小。增加了一张图,
它后面图的标号又得重新来编一次,还有图的引用处也不要忘记……增加了一段文字,发现排好
的图又跑到下面一页去了。。。。。这一页留下了好大一片空白。下一页的表格又被它推
成了分布在两页。参考文献引用增加或者删除了一个,那就得找出全文要改动的所有引用处!!
什么,你是高级用户,会使用交叉引用,好吧,那有时候打出来的文档突然出现“错误!找不到引用--”
这样的字样,你是什么心情?
生成目录字体不一样,有的标题起的名太长,把页码挤到下一行去了,一个个改好了,一全更新又完了。
如果你的文档里还有一些公式,那又是一场战争了,有的公式字显得比正文大,有的又显得比正文小,
或者总是跟写的编号对不齐,不是偏上就是偏下。公式编辑器\index{公式编辑器}版本众多,换台电脑就只能看不能改了,
公式编辑器还经常提示已过期。blablabla……

千辛万苦,终于搞定了,看起来还算漂亮,到打印店去了,omg,word版本不对,版白排了,又乱了。
用PDF吧,不同软件生成的PDF总是跟原来的样子有点差距。做完论文,感觉是扒了一层皮。

\LaTeX{}解决了这个问题,它实际可以看成是一种写给机器的语言,把格式定义好,再填上内容,
它就会按设定好的格式把一份文档生成出来,一般是生成pdf文档,整个文档的格式首先是统一的,
不会出现不同章节的显示不一样的问题,并且,文档的格式可以是封装好的,使用的时候不需要
理会具体格式,直接可以使用,封闭格式这个功能留给会的人去做,写论文不需要关心太多的格式问题。

如果是往国外投文章,那么\LaTeX{}的应用更广泛,不少国外出版社直接提供其格式文件,
只要将文档头上的$\backslash$documentclass\{article\} 大括号里的“article”换成其相应的格式文件即可,
避开了格式调整问题。

话说回来,Word还是有优势的,所见即所得,上手容易,会鼠标键盘就会用。\LaTeX{}还是需要一些入门时间的,
如果单纯会使用这个模版,我想大约需要3个小时左右,如果会处理一些常见的程序错误以及做小的格式修改,
时间就会比较长了,大约要几天,有人引导的话会快一些。
Word在做小文档,比如就几页的文档上的优势\LaTeX{}{}是无法与之比拟的,做起来是很快,比如通知啦,传单啦。
但Word存在的意义应该不是就为了那几张小广告的,几百M的体积,几百RMB的价格,当然盗版很多。在这种应用
上还不如去用一下免费的WPS,或者OpenOffice。Word支持很多特性,支持宏,支持Visual Basic……,但是,它们
又太难了,不是随便一个人就会去有兴趣学它们的,学了很少有机会用到,屠龙之技。

\LaTeX{}{}还有两个优势是Word所没有的:
\begin{enumerate}
\item{文件体积小}

使用\LaTeX{}{},所要编辑的文件以“tex”为扩展名,如果用到参考文献,可能还需要扩展名为“bib”的参考文献数据库
文件,此外,还可能有的就是文档中要插入的图片文件了。
“tex”和“bib”文件都是纯文本文件,如果愿意,可以用记事本来编辑,按照一定的格式书写,格式也是很简单的,
看到了就会使用。
而Word文件是Microsoft自己定义的二进制文件,只有用Word软件或者其它兼容的软件打开,因为是Microsoft自己的二进制格式,
因此,体积会比较大,当然文件集成度很好,只有一个文件。
相比较之下\LaTeX{}{}可能有很多文件,但这个缺陷完全可以通过压缩软件打包来完成。
如果打开一个比较大的文档,机器破的话会比较慢,而且容易出一些错误导致Word意外关闭,想来各位都遇到过这种情况的吧。
文件如果发生了意外关闭,那就有可能被损坏,损坏后就有可能。。。。。打不开了,如果没有做一些备份,
那就成了“杯具”甚至“餐具”了。相较而言,\LaTeX{}{}的文件小,而且是纯文本文件,即使被损坏,
修复起来也比Word要容易得多。

\item{使写作更加专注于内容}

说实话,这一条在学会使用\LaTeX{}{}之前,我觉得是扯淡,那时的我觉得,使用Word边写边想也一样很快的。但在我学习使用\LaTeX{}{}
的过程中,我逐渐感受到,当只面对文本,不去想它下一段怎么排,什么样的格式时,思维更加连续,写起来也更快,
而且明显感觉到自己进入了一种写作的状态,这种感觉只在以前纸上写作感觉得到,专注于自己想的内容。
这一点,只有在学会使用之后,才能去体会得到。

\end{enumerate}


\section{模板说明}

ZJUTThesis~是为了帮助浙江工业大学本科毕业生撰写毕业论文而编写的~\XeLaTeX~论文模板,其前提是用户已经能处理一般的~\XeLaTeX~文档,并对~BibTeX~有一定了解,如果你从来没有接触过~\XeLaTeX~,建议先学习相关基础知识,磨刀不误砍柴工,能有助你更好使用模板\cite{huwei}。

由于作者水平有限,虽然现在的这个版本基本上满足了学校的要求,但难免存在不足之处,欢迎大家积极反馈,更希望浙江工业大学~\XeLaTeX~爱好者能一同完善此模板,让更多同学受益。

如有模板的疑问或有意向加入模板的维护和编写队伍中来,请给作者:Monster(diufanshu@gmail.com),或09级版本作者 Unlucky(unlucky1990@gmail.com)或MCKelvin(ibmmc@live.com)写信。


理论上,并不一定要把每章放在不同的文件中。但是这种自顶向下,分章节写作、编译的方法有利于提高效率,大大减少~Debug~过程中的编译时间,同时减小风险。

\section{下载安装}
本分支主页:~\url{https://github.com/diufanshu/zjutthesis}。除此之外,不再维护任何镜像。


\section{参考文献生成方法}

\LaTeX~具有插入参考文献的能力。Google Scholar~网站上存在兼容~BibTeX~的参考文献信息,通过以下几个步骤,可以轻松完成参考文献的生成。
\begin{itemize}
  \item 在\href{http://scholar.google.com/}{谷歌学术搜索}中,
        点击\href{http://scholar.google.com/scholar_preferences?hl=en&as_sdt=0,5}{学术搜索设置}。
  \item 页面打开之后,在\textbf{文献管理软件}选项中选择\textbf{显示导入~BibTeX~的链接},单击保存设置,退出。
  \item 在谷歌学术搜索中检索到文献后,在文献条目区域单击导入~BibTeX~选项,页面中出现文献的引用信息。
  \item 将文献引用信息的内容复制之后,添加到~references~文件夹下的~reference.bib~中。
\end{itemize}

\section{编译注意事项}
由于模板使用~UTF-8~编码,所以源文件应该保存成~UTF-8~格式,否则可能出现中文字符无法识别的错误。
  本模板中每一个~.tex~文件的文件的开头已经加上一行:\\
  \verb|% !Mode:: "TeX:UTF-8"|\\
     这样可以确保~.tex~文件默认使用~UTF-8~的格式打开。读者如果删去此行,很有可能会导致中文字符显示乱码。
     在~WinEdt~编辑器中可以使用以下两种方式保存成~UTF-8~格式

\section{系统要求}
    CTEX 2.9, MiKTeX 2.9 或者 TeX Live 2013。当然,新版本也是可以用的,注意宏包的更新。默认使用推荐的~WinEdt 7.0~编辑器可以完成文件的编辑和编译工作,或者直接用记事本也行,我用的是ST3+latextool,但这是盗版,需要面壁。。。

\section{\TeX~简介}

以下内容是~milksea@bbs.ctex.org~撰写的关于~\TeX~的简单介绍,略有改动。
注意这不是一个入门教程,不讲~\TeX~系统的配置安装,也不讲具体的~\XeLaTeX~代码。
这里仅仅试图以一些只言片语来解释:
进入这个门槛之前新手应该知道的注意事项,以及遇到问题以后该去如何解决问题。

\subsection{什么是 \TeX/\XeLaTeX,我是否应该选择它~?}

\TeX~是最早由高德纳(Donald Knuth)教授创建的一门标记式宏语言,
用来排版科技文章,尤其擅长处理复杂的数学公式。\TeX~同时也是处理这一语言的排版软件。
\XeLaTeX~是 Leslie Lamport 在~\TeX~基础上按内容/格式分离和模块化等思想建立的一集~\TeX~上的格式。

\TeX~本身的领域是专业排版领域
但现在~TeX/LaTeX~也被广泛用于生成电子文档甚至幻灯片等,~\TeX~语言的数学部分
偶尔也在其他一些地方使用。但注意~\TeX~并不适用于文书处理(Microsoft Office 的领域,以前和现在都不是)。

选择使用~\TeX/\XeLaTeX~的理由包括:
\begin{itemize}
\item 免费软件;
\item 专业的排版效果;
\item 是事实上的专业数学排版标准;
\item 广泛的西文期刊接收甚或只接收 LaTeX 格式的投稿;
\item[] ……
\end{itemize}
不选择使用~\TeX/\XeLaTeX~的理由包括:
\begin{itemize}
\item 需要相当精力学习;
\item 图文混合排版能力不够强;
\item 仅在数学、物理、计算机等领域流行;
\item 中文期刊的支持较差;
\item[] ……
\end{itemize}

请尽量清醒看待网上经常见到的关于~\TeX~与其他软件的优劣比较和口水战。在选择使用或离开之前,请先考虑
\TeX~的应用领域,想想它是否适合你的需要。


\subsection{我该用什么编辑器~?}

编辑器功能有简有繁,特色不一,从简单的纯文本编辑器到繁复的 Emacs,因人而易。基本功能有语法高亮、方便编译预览就很好了,扩充功能和定制有无限的可能。初学者可以使用功能简单、使用方便的专用编辑器,如 ~TeXWorks、Kile、WinEdt~等,或者类似所见即所得功能的~LyX;熟悉的人可以使用定制性更强的~Notepad++、SciTE、Vim、Emacs ~等。这方面的介绍很多,一开始不妨多试几种,找到最适合自己的才是最好的。

另外提醒一句,编辑器只是工作的助手,不必把它看得太重。

\subsection{我应该看什么~\XeLaTeX~读物~?}

这不是一个容易回答的问题,因为有许多选择,也同样有许多不合适的选择。
这里只是选出一个比较好的答案。更多更详细的介绍可以在版面和网上寻找(注意时效)。

近两年~\TeX~的中文处理发展很快,目前没有哪本书在中文处理方面给出一个最新进展的合适综述,
因而下面的介绍也不主要考虑中文处理。

\begin{enumerate}

\item 我能阅读英文。
\begin{enumerate}
\item 迅速入门:ltxprimer.pdf (LaTeX Tutorials: A Primer, India TUG)
\item 系统学习:A Guide to LaTeX, 4th Edition, Addison-Wesley
               有机械工业出版社的影印版(《\LaTeX{}~实用教程》)
\item 深入学习:要读许多书和文档,TeXbook 是必读的
\item 细节学习:去读你使用的每一个宏包的说明文档
\item 专题学习:阅读讲数学公式、图形、表格、字体等的专题文档
\end{enumerate}

\item 我更愿意阅读中文。
\begin{enumerate}
\item 迅速入门:lnotes.pdf (LaTeX Notes, 1.20, Alpha Huang)
\item 系统学习:《\LaTeXe{}~科技排版指南》,邓建松(电子版)
      如果不好找,可以阅读《\LaTeXe~入门与提高》第二版,陈志杰等,或者 《\LaTeXe~完全学习手册》,胡伟
\item 深入学习:~TeXbook0.pdf~(特可爱原本,TeXbook 的中译,xianxian)
\item 具体问题释疑:~CTeX-FAQ.pdf~,\\
        吴凌云,~\url{http://www.ctex.org/CTeXFAQ}~
\end{enumerate}
\end{enumerate}

遇见问题和解决问题的过程可以快速提高自己的技能,建议此时:
\begin{itemize}
  \item 利用~Google~搜索。
  \item 清楚,扼要地提出你的问题。
\end{itemize}

\subsection{什么知识会过时~?什么不会~?}

\TeX~是排版语言,也是广泛使用的软件,并且不断在发展中;
因此,总有一些东西会很快过时。作为学习~\TeX~的人,
免不了要看各种各样的书籍、电子文档和网络论坛上的只言片语,
因此了解什么知识会迅速过时,什么知识不会是十分重要的。

最稳定的是关于~Primitive \TeX~和~Plain \TeX~的知识,也就是 Knuth
在他的《The TeXbook》中介绍的内容。因为~\TeX~
系统开发的初衷就是稳定性,要求今天的文档到很久以后仍可以得到完全相同的结果,
因此 Knuth 限定了他的~\TeX~语言和相关实现的命令、语法。这些内容许多年来就没有多少变化,
在未来的一些年里也不会有什么变化。
Primitive \TeX~和 Plain \TeX~的知识主要包括 \TeX~排版的基本算法和原理,
盒子的原理,底层的 \TeX~命令等。其中技巧性的东西大多在宏包设计中,
初学者一般不会接触到很多;而基本原理则是常常被提到的,
譬如,~\TeX~把一切排版内容作为盒子(box)处理。

相对稳定的是关于基本~\LaTeXe~
的知识,也包括围绕~\LaTeXe~的一些核心宏包的知识。
在可预见的将来,~\LaTeXe~不会过时。
\LaTeXe~的知识是目前大部分~\LaTeX~书籍的主体内容。关于~\XeLaTeX~的标准文档类
~(article、report、book、letter、slide~等),关于基本数学公式的输入,
文档的章节层次,表格和矩阵,图表浮动体,LR 盒子与段落盒子……
这些~\XeLaTeX~的核心内容都是最常用的,相对稳定的。
与~\LaTeXe~相匹配的核心宏包,
如~graphics(x)、ifthen、fontenc、doc~等,也同样是相对稳定的。
还有一些被非常广泛应用的宏包,如~amsmath~系列,也可以看作是相对稳定的。

简单地说,关于基本~\TeX/\XeLaTeX~的语言,都是比较稳定的。与之对应,实现或者支持~\TeX/\XeLaTeX~语言的软件,
包括在~\TeX/\XeLaTeX~基础上建立的新的宏,都不大稳定。

容易过时的是关于第三方~\XeLaTeX~宏包的知识、第三方~\TeX~工具的知识,以及新兴~\TeX~相关软件的知识等。
~\TeX~和~\XeLaTeX~语言是追求稳定的;但无论是宏包还是工具,作为不断更新软件,它们是不稳定的。
容易过时的技术很多,而且现在广泛地出现在几乎所有~\XeLaTeX~文档之中,因此需要特别引起注意:
宏包的过时的原因可能是宏包本身的升级换代带来了新功能或不兼容,
也可能是同一功能的更新更好的宏包代替了旧的宏包。前者的典型例子比如绘图宏包~PGF/TikZ~,
现在的~2.00~版功能十分强大,和旧的~1.1x~版相差很大,和更旧的~0.x~版本则几乎完全不同;后
者的典型例子比如~caption~宏包先是被更新的~caption2~宏包代替,后来~caption~宏包更新又使得
caption2 宏包完全过时。——安装更新的发行版可以避免使用过旧的宏包;
认真阅读宏包自带的文档而不是搜索得到的陈旧片断可以避免采用过时的代码。

工具过时的主要原因也是升级换代和被其他工具替换。前者的典型例子是编辑器
WinEdt~在~5.5~以后的版本支持~UTF-8~编码,而旧版本不支持;
后者的典型例子是中文字体安装工具从~GBKFonts~到~xGBKFonts~到~FontsGen~不断被取代。
图形插入是一个在~\TeX~实现、宏包与外围工具方面都更新很快的东西。
在过去,最常用的输出格式是~PS(PostScript)~格式,因此插入的图像以~EPS~为主流。
使用~Dvips~为主要输出工具,外围工具有~GhostScript、bmeps~等等,相关宏包有~graphics~等,
相关文档如《\LaTeXe{}~ 插图指南》。

\XeLaTeX~不限定图片格式,推荐使用EPS格式的图片,但是PNG和JPEG格式的图片也支持。

值得特别提出注意的就是,中文处理也一起是更新迅速、容易过时的部分。
而且因为中文处理一直没有一个“官方”的“标准”做法,软件、工具、
文档以及网上纷繁的笔记也就显得相当混乱。从八十年代开始的~CCT~系统、
天元系统,到后来的~CJK~方式,到近来的~XeTeX~和~LuaTeX~ 方式,
中文处理的原理、软件、宏包、配置方式等都在不断变化中。

\section{免责声明}

本模板依据《浙江工业大学本科生毕业设计说明书(论文)模板》编写,作者希望能给使用者写作论文带来方便。然而,作者不保证本模板完全符合学校要求,也不对由此带来的风险和损失承担任何责任。